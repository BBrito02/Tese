%!TEX root = ../template.tex
%%%%%%%%%%%%%%%%%%%%%%%%%%%%%%%%%%%%%%%%%%%%%%%%%%%%%%%%%%%%%%%%%%%%
%% abstract-en.tex
%% NOVA thesis document file
%%
%% Abstract in English([^%]*)
%%%%%%%%%%%%%%%%%%%%%%%%%%%%%%%%%%%%%%%%%%%%%%%%%%%%%%%%%%%%%%%%%%%%

\typeout{NT FILE abstract-en.tex}%

Nowadays, it has become increasingly necessary to process and analyze large volumes of data. The use of tools such as dashboards for analysis and decision support in the context of interactive data visualization has become essential. Dashboards oriented towards interactive data visualization aim primarily to address the various analytical questions defined by their users. These questions are answered through the use of visualization techniques and interaction taxonomies to represent information clearly and objectively. While visualization techniques allow data to be represented visually, interaction taxonomies help users explore the presented information from different perspectives. However, the process of creating dashboards often requires an iterative approach, demanding constant communication between designers and stakeholders. This communication process becomes quite complex and is sometimes overlooked due to the lack of support tools at this prototyping stage. As a response to this gap, \gls{IVML} emerged as a proposal for documenting and structuring the different components that make up a dashboard. The goal of this dissertation is to develop a tool based on the \gls{IVML} language, capturing all stages of dashboard creation—from the definition of the different analytical scopes to the implementation and final documentation of the visualizations and interactions in the finished dashboard. This tool will facilitate communication among the stakeholders in the design process, streamlining the entire creation and prototyping workflow. The tool will be implemented through the development of different prototypes and meta-models, which will be tested and validated to ensure their effectiveness and utility. With this approach, the entire dashboard creation process is expected to be optimized, enabling better adaptation to user needs and greater efficiency in the implementation of the proposed solutions.

% Palavras-chave do resumo em Inglês
% \begin{keywords}
% Keyword 1, Keyword 2, Keyword 3, Keyword 4, Keyword 5, Keyword 6, Keyword 7, Keyword 8, Keyword 9
% \end{keywords}
\keywords{
  Interactive Data Visualization \and
  Dashboard Design \and
  IVML Designer \and
  Data Analysis 
}
