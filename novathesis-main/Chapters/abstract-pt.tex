%!TEX root = ../template.tex
%%%%%%%%%%%%%%%%%%%%%%%%%%%%%%%%%%%%%%%%%%%%%%%%%%%%%%%%%%%%%%%%%%%%
%% abstract-pt.tex
%% NOVA thesis document file
%%
%% Abstract in Portuguese
%%%%%%%%%%%%%%%%%%%%%%%%%%%%%%%%%%%%%%%%%%%%%%%%%%%%%%%%%%%%%%%%%%%%

\typeout{NT FILE abstract-pt.tex}%

Atualmente, tem vindo a ser cada vez mais necessário processar e analisar grandes volumes de dados. A utilização de ferramentas como \textit{dashboards} para análise e apoio à tomada de decisão no contexto da visualização interativa de dados tem-se tornado algo essencial. As \textit{dashboards} orientadas para a visualização interativa de dados apresentam como principal objetivo, responder às diversas questões analíticas definidas por parte dos seus utilizadores. Estas questões são respondidas através da utilização de técnicas de visualização e taxonomias de interação para representar a informação de forma clara e objetiva. Enquanto que as técnicas de visualização permitem representar os dados a serem analisados de forma visual, as taxonomias de interação ajudam os utilizadores a explorar a informação apresentada sob diferentes perspetivas. No entanto, o processo de criação de \textit{dashboards} requer uma abordagem frequentemente iterativa, exigindo uma constante comunicação entre \textit{designers} e \textit{stakeholders}. Esta processo de comunicação torna-se uma tarefa bastante complicada e por vezes omitida devido à falta de ferramentas de suporte nesta fase de prototipagem. Como resposta a esta lacuna, surgiu a \gls{IVML}, uma proposta de documentação e estruturação das diferentes componentes que integram um \textit{dashboard}. O objetivo desta dissertação é a criação de uma ferramenta que tenha a linguagem \gls{IVML} como base, onde iram ser captadas todas as fases de criação de um \textit{dashboard}, desde a definição dos diferentes âmbitos de análise, até à implementação e documentação final das visualizações e interações presentes no \textit{dashboard} final. Esta ferramenta irá facilitar a comunicação entre os seus intervenientes no processo de \textit{design}, facilitando todo o processo de criação e prototipagem. A ferramenta irá ser implementada através da criação de diferentes protótipos e meta-modelos, que irão ser testados e validados com o objetivo de garantir a sua eficácia e utilidade. Com esta abordagem, é expectável que todo o processo de criação de \textit{dashboards} seja otimizado, permitindo uma melhor adaptação às necessidades dos utilizadores e uma maior eficiência na implementação das soluções propostas.

% Palavras-chave do resumo em Português
% \begin{keywords}
% Palavra-chave 1, Palavra-chave 2, Palavra-chave 3, Palavra-chave 4
% \end{keywords}
\keywords{
  Visualização Interativa de Dados \and
  Design de Dashboards \and
  IVML Designer \and
  Análise de Dados 
}
% to add an extra black line