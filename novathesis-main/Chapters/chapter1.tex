%!TEX root = ../template.tex
%%%%%%%%%%%%%%%%%%%%%%%%%%%%%%%%%%%%%%%%%%%%%%%%%%%%%%%%%%%%%%%%%%%
%% chapter1.tex
%% NOVA thesis document file
%%
%% Chapter with introduction
%%%%%%%%%%%%%%%%%%%%%%%%%%%%%%%%%%%%%%%%%%%%%%%%%%%%%%%%%%%%%%%%%%%

\typeout{NT FILE chapter1.tex}%

\chapter{Introdução}
\label{cha:introducao}

\section{Contexto e Motivação}
\label{sec:cont_e_motiv}

Aqui é onde eu escrevo a introdução do meu trabalho.  Esta é a primeira parte do texto que o leitor vai ler, e é importante que seja bem escrita e clara.  A introdução deve apresentar o tema do trabalho, a motivação para o estudo, e o que o leitor pode esperar encontrar no resto do documento.

\section{Abordagem (Problema e Abordagem)}
\label{sec:abordagem}

\section{Contribuições}
\label{sec:contribuicoes}

\section{Organização do Documento}
\label{sec:organizacao}

