%!TEX root = ../template.tex
%%%%%%%%%%%%%%%%%%%%%%%%%%%%%%%%%%%%%%%%%%%%%%%%%%%%%%%%%%%%%%%%%%%
%% chapter1.tex
%% NOVA thesis document file
%%
%% Chapter with introduction
%%%%%%%%%%%%%%%%%%%%%%%%%%%%%%%%%%%%%%%%%%%%%%%%%%%%%%%%%%%%%%%%%%%

\typeout{NT FILE chapter1.tex}%

\chapter{Introdução (SOB DESENVOLVIMENTO)}
\label{cha:introducao}

Neste capítulo introdutório, irão ser apresentados os elementos crucias que sustentam e enquadram o trabalho desenvolvido ao longo da dissertação. Inicialmente, será feita uma breve contextualização do tema abordado bem como a motivação para o seu estudo. De seguida, será apresentada uma descrição do problema abordado, bem como os objetivos e as questões de investigação que guiaram o trabalho. Por fim, será feita uma breve descrição da organização do documento.

\section{Contexto e motivação}
\label{sec:cont_e_motiv}

A necessidade de processar e apresentar informação de forma clara, eficiente e intuitiva tem vindo a crescer com o passar dos anos devido a um crescimento exponencial na geração de dados. Desta forma, a introdução de ferramentas que sirvam de suporte para a interpretação e visualização de grandes volumes de dados, e que ajudem na tomada de decisão tornam se cada vez mais necessárias. Neste contexto, os \textit{dashboards} surgem como uma solução para a visualização de informação, permitindo a sua representação de forma clara, através de diferentes representações visuais e mecanismos de interação, respondendo a questões analiticas impostas pelos seus usuários. Os \textit{dashboards} proporcionam uma visão clara e objetiva dos dados, facilitando a identificação de padrões, tendências e anomalias. No entanto, a sua criação e implementação pode se tornar um desafio. Muitas vezes, os seus \textit{designers} apresentam como foco principal a representação do máximo de informação possível, sobrecarregando o utilizador, em vez de se focarem na eficácia e utilidade da informação apresentada.

Para que o desenvolvimento de \textit{dashboards} seja feito de forma correta, é necessário que exista um diálogo constante entre o \textit{stakeholders} e o \textit{designer}. É através deste diálogo que conseguimos garantir que os \textit{dashboards} implementados satisfaçam os objetivos e necessidades do seu público-alvo. No entanto, a ausência de ferramentas de prototipagem dificulta bastante este processo de desenvolvimento, levando a atrasos e até mesmo a falhas na sua implementação. Embora existam várias ferramentas, como o \textit{Tableau} e o \textit{Power BI}, estas ferramentas concentram as suas funcionalidades na visualização e análise de dados, deixando a fase de \textit{design} e prototipagem de parte. Existe assim, uma grande motivação para o desenvolvimento de uma ferramenta que apoie o \textit{design} sistemático de \textit{dashboards}, permitindo a prototipagem e documentação de forma clara e intuitiva.

É neste contexto que surge a \gls{IVML} \cite{Ferreira2023IVML}, uma proposta de linguagem de modelação e prototipagem para a visualização interativa de dados. A \gls{IVML} foi desenvolvida com o objetivo de colmatar a lacuna existente na fase de conceção e prototipagem que maior parte das ferramentas atuais enfrentam. A \gls{IVML} foca-se na documentação e modelação de \textit{dashboards} interativos, permitindo representar as diferentes visualizações, interações e elementos visuais que iram ser utilizados. Quando é apresentado um \textit{dashboard} a um utilizador, é importante que este consiga compreender de forma clara e intuitiva os diferentes elementos presentes, bem como a sua intenção no \textit{dashboard}. A \gls{IVML} permite a descrição de forma simples, formal, e intuitiva das intenções de cada componente presente no \textit{dashboard}, facilitando a percepção e utilização da mesma. A documentação e representação dos diferentes elementos presentes num \textit{dashboard} tornam a comunicação entre \textit{stakeholders} e \textit{designers} uma tarefa mais fácil e intuitiva, promovendo a sua correta implementação e desenvolvimento.

%É neste contexto que surge a \gls{IVML} (CITAR IVML), uma linguagem visual dedicada à modelação e documentação de dashboards interativos. A \gls{IVML} foi desenvolvida com o objetivo de colmatar a lacuna existente na fase de conceção, oferecendo uma forma formal de representar os diversos componentes de um dashboard, como visualizações, interações e tooltips. Esta linguagem permite descrever de forma clara as intenções de design, facilitando a comunicação entre \textit{designers} e \textit{stakeholders} promovendo um processo de desenvolvimento mais ágil, iterativo e centrado no utilizador.
\section{Descrição do problema}
\label{sec:des_problema}

Os \textit{dashboards} desempenham um papel fundamental no que toca a visualização e análise de dados, sendo através das diferentes representações visuais e mecanismos de interação presentes que os utilizadores vêm respondidas as suas questões analíticas. No entanto, como é referido na secção \ref{sec:cont_e_motiv}, o processo de desenvolvimento de \textit{dashboards} é um processo complexo e desafiante. Para que esta implementação seja bem-sucedida, é necessário ter em consideração vários fatores:

\begin{itemize}
    \item Conhecer o público-alvo;
    \item Definir o propósito e os objetivos;
    \item Selecionar e preparar os dados relevantes;
    \item Focar na clareza e eficácia da comunicação;
\end{itemize}

O processo de desenvolvimento de \textit{dashboards} é um processo interativo, onde a constante comunicação entre os \textit{stakeholders} e os \textit{designers} é fundamental para garantir que o produto final satisfaça as necessidades e expectativas do público-alvo. Este diálogo é muitas vezes atrasado, e por vezes omitido, uma vez que existe uma falta de ferramentas que suportem esta fase inicial de \textit{design} e prototipagem. Por vezes, este diálogo só é realizado numa fase final, onde já se encontram implementados os diferentes \textit{dashboards}. Esta falta de comunicação pode resultar numa implementação falhada, onde os \textit{dashboards} não satisfazem as necessidades do seu público-alvo, o que resulta num desperdício de tempo e recursos.

Neste contexto, surge a necessidade de uma ferramenta que suporte o processo de \textit{design} e prototipagem de \textit{dashboards}, uma vez que de momento não existe nenhuma ferramenta de suporte que os \textit{designers} possam utilizar para mostrar aos utilizadores como será o desenvolvimento e produto final da \textit{dashboad}, nomeadamento, que visualizações iram ser usadas, qual a estrutura e organização optada, que interações vão estar disponíveis e quais os seus impactos. A linguagem \gls{IVML} foi criada com o principal intuito de preencher essa lacuna, oferecendo um modelo capaz de representar as diferentes componentes de um \textit{dashboard}, facilitando assim a comunicação entre os \textit{stakeholders} e os \textit{designers}, uma vez que a linguagem oferce uma documentação detalhada de todas as componentes e interações presentes no seu \textit{design}.

\section{Abordagem e contributos esperados}
\label{sec:contribuicoes}

% Basear me no documento de resumo que fiz sobre a reunião de dia 05/27/2025


\section{Organização do documento}
\label{sec:organizacao}

Neste capítulo irá ser descrita a orgaização do documento de forma a facilitar a leitura e compreensão do mesmo. 

No capítulo atual é feita uma contextualização do tema da dissertação, dando ênfase à crescente necessidade de ferramentas para o design de \textit{dashboards} interativos (\ref{sec:cont_e_motiv}). É identificado e apresentado o problema atual no desenvolvimento de \textit{dashboards}, refletido pela falta de ferramentas de apoio à prototipagem e comunicação entre \textit{stakeholders} e \textit{designers} (\ref{sec:des_problema}). (FALTA FALAR SOBRE A ABORDAGEM E CONTRIBUTOS ESPERADOS) 

No capítulo \ref{cha:conceitos_base} são abordados os coiceitos teóricos que suportam esta dissertação. O capítulo é iniciado pela apresentação dos conceitos relacionados com a visualização de dados (\ref{sec:vis_dados}), abordando os seus princípios (\ref{sub:fundamentos}), o papel que a interação desepenha neste tipo de sistemas (\ref{sub:elem_int}), terminando pela apresentação das diferentes taxonomias de interação que suportam a visualização de dados (\ref{sub:taxonomias_interacao}). No capítulo seguinte, é feita uma introdução ao \textit{design} de \textit{dashboards} (\ref{cha:design_dashboard}), onde são abordadas as recomendações e boas práticas a ter em consideração durante o seu desenvolvimento (\ref{sub:recomendacoes_design}), bem como os desafios e limitações impostos (\ref{sub:desafios_design}), os diferentes tipos de \textit{dashboards} e as suas caracteristicas (\ref{sub:tipos_dashboards}), terminando pela listagem e análise das diferentes ferramentas de desenho de \textit{dashboards} existentes e as suas limitações (\ref{sub:ferramentas}). De forma semelhante, no capítulo seguinte é feita uma introdução ao \textit{design} de interfaces de utilizador (\ref{cha:ui_design}), onde são introduzidos os principios de usabilidade e experiência do utilizador em interfaces interativas (\ref{sub:principios_usabilidade}). No capítulo seguinte é introduzido o conceito de avaliação da usabilidade de linguagens visuais (\ref{cha:avaliacao_usabilidade}), onde é abordado o conceito teórico \textit{Physics of Notation} e os princípios que suportam esta teroia descritiva (\ref{sub:physics_notation}). Por fim, é apresentada a linguagem IVML, onde é feita a sua descrição ao nível sintático e semântico (\ref{cha:esp_ivml}).

No capítulo \ref{cha:ivml_designer} vai ser apresentado o IVML Designer. Inicialmente irá ser exibido o meta-modelo desenvolvido para a linguagem IVML (CONTINUAAR QUANDO IMPLEMENTAR DE FACTO O CAPITULO)

%Nocapítulo 3 é apresentado o IVML Designer. Começa-se por mostrar o meta-modelo desenvolvido para a linguagem IVML, especificando quais os elementos que irão fazer parte da ferramenta bem como todas as decisões tomadas durante o período anterior ao desenvolvimento.