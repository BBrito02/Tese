%!TEX root = ../template.tex
%%%%%%%%%%%%%%%%%%%%%%%%%%%%%%%%%%%%%%%%%%%%%%%%%%%%%%%%%%%%%%%%%%%
%% chapter1.tex
%% NOVA thesis document file
%%
%% Chapter with introduction
%%%%%%%%%%%%%%%%%%%%%%%%%%%%%%%%%%%%%%%%%%%%%%%%%%%%%%%%%%%%%%%%%%%

\typeout{NT FILE chapter1.tex}%

\chapter{Introdução}
\label{cha:introducao}

Neste capítulo introdutório, irão ser apresentados os elemetnos crucias que sustentam e enquadram o trbalho desenolvido ao longo da dissertação. Inicialmente, será feita uma breve contextualização do tema abordado bem como a motivação para o seu estudo. De seguida, será apresentada uma descrição do problema abordado, bem como os objetivos e as questões de investigação que guiaram o trabalho. Por fim, será feita uma breve descrição da organização do documento.

\section{Contexto e Motivação}
\label{sec:cont_e_motiv}

\section{Descrição do Problema}
\label{sec:des_problema}

\section{Trabalho Realizado e Contributos}
\label{sec:contribuicoes}

\section{Organização do Documento}
\label{sec:organizacao}

