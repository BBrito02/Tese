%!TEX root = ../template.tex
%%%%%%%%%%%%%%%%%%%%%%%%%%%%%%%%%%%%%%%%%%%%%%%%%%%%%%%%%%%%%%%%%%%
%% chapter1.tex
%% NOVA thesis document file
%%
%% Chapter with introduction
%%%%%%%%%%%%%%%%%%%%%%%%%%%%%%%%%%%%%%%%%%%%%%%%%%%%%%%%%%%%%%%%%%%

\typeout{NT FILE chapter1.tex}%

\chapter{Introdução}
\label{cha:introducao}

Neste capítulo introdutório, irão ser apresentados os elementos cruciais que sustentam e enquadram o trabalho desenvolvido ao longo da dissertação. Inicialmente, será feita uma breve contextualização do tema abordado bem como a motivação para o seu estudo. De seguida, será feita a descrição do problema identificado. Por fim, será feita uma breve descrição da organização do documento.

\section{Contexto e motivação}
\label{sec:cont_e_motiv}

A necessidade de processar e apresentar informação de forma clara, eficiente e direta, tem vindo a crescer com o passar dos anos devido a um crescimento exponencial da geração de dados \cite{kitchin2014data}. Este aumento exponencial tem motivado o aparecimento de \textit{dashboards} analíticos, e de ferramentas dedicadas ao seu desenho e prototipagem que sirvam de suporte para a tomada de decisão. Este tipo de \textit{dashboards} são desenvolvidos com o principal objetivo de dar apoio à exploração e análise aprofundada da informação que, através da sua elevada capacidade de interatividade, promovem a análise e interpretação de grandes volumes de dados. É através da combinação das diferentes técnicas de interação com as respetivos componentes visuais que os utilizadores conseguem explorar os dados e obter respostas para as suas questões analíticas. O papel que as diferentes componentes de interação desempenham neste tipo de ferramentas é crucial, uma vez que é dada a possibilidade de o utilizador explorar a informação livremente, resultando numa maior retenção de informação em comparação com visualizações estáticas \cite{brucker2014learning}. Os \textit{dashboards} proporcionam uma visão simples e objetiva dos dados, facilitando a identificação de padrões, tendências e anomalias \cite{few2006information}. No entanto, a sua criação e implementação pode tornar-se um desafio. Muitas vezes, os seus \textit{designers} optam por representar o máximo de informação possível, sobrecarregando o utilizador, em vez de optarem por uma abordagem que que valorize a eficácia e utilidade da informação apresentada \cite{eckerson2010performance}.

Para que o desenvolvimento de \textit{dashboards} seja feito de forma correta, é necessário que exista um diálogo constante entre os seus utilizadores. É através deste diálogo que conseguimos garantir que os \textit{dashboards} implementados satisfaçam os objetivos e necessidades do seu público-alvo. No entanto, a ausência de ferramentas de prototipagem dificulta bastante este processo de desenvolvimento, levando a atrasos e até mesmo a falhas na sua implementação. Embora existam várias ferramentas, como o \textit{Tableau} \cite{tableau} e o \textit{Power BI} \cite{powerBI}, estas ferramentas apenas concentram as suas funcionalidades na implementação e construção de \textit{dashboards}, omitindo todo o processo defendido pelos princípios de \gls{HCI} que estipulam que uma aplicação deve sempre passar por um processo cíclico em que são desenvolvidos protótipos, recolhidos o \textit{feedback} dos utilizadores e feitas iterações sucessivas de melhoria com base nessa avaliação feitas pelos seus usuários. Existe assim uma grande motivação para o desenvolvimento de uma ferramenta que apoie o design sistemático de \textit{dashboards}, onde seja possível realizar protótipos e documentar de forma clara as diferentes alternativas de visualização e organização dos dados. Esta ferramenta tem como objetivo facilitar a recolha de \textit{feedback} por parte dos utilizadores, possibilitar a simulação do comportamento interativo do dashboard e apoiar a comunicação entre os seus intervenientes ao longo de todas as fases de conceção.

É neste contexto que surge a \gls{IVML} \cite{Ferreira2023IVML}, uma proposta de linguagem de modelação gráfica e prototipagem orientada para a visualização interativa de dados. A \gls{IVML} foi desenvolvida como uma primeira tentativa de colmatar a lacuna existente na fase de conceção e prototipagem que maior parte das ferramentas atuais enfrentam. A \gls{IVML} foca-se na modelação e documentação de \textit{dashboards} interativos, permitindo representar as diferentes visualizações, interações, navegações e elementos visuais que irão ser utilizados. Ao ser apresentado um \textit{dashboard} a um utilizador, é importante que este consiga compreender os diferentes elementos presentes, bem como a sua intenção sem ter de apresentar um esfroço a nível cognitivo significante. A \gls{IVML} permite a descrição de forma gráfica e formal das intenções de cada componente presente no \textit{dashboard}, facilitando a compreensão e utilização da mesma. A documentação e representação dos diferentes elementos presentes num \textit{dashboard} tornam a comunicação entre os seus intervenientes uma tarefa simples e direta, promovendo a sua correta implementação e desenvolvimento. Apesar do \gls{IVML} constituir um contributo muito relevante, a linguagem de forma isolada não resolve de forma completa o problema da prototipagem.

\section{Descrição do problema}
\label{sec:des_problema}

Os \textit{dashboards} desempenham um papel fundamental no que toca a visualização e análise de dados, sendo através das diferentes representações visuais e mecanismos de interação presentes que os utilizadores vêm respondidas as suas questões analíticas. No entanto, como é referido na secção \ref{sec:cont_e_motiv}, o processo de desenvolvimento de \textit{dashboards} é um processo complexo e desafiante. %Para que esta implementação seja bem-sucedida, é necessário ter em consideração vários fatores:
\begin{comment}
\begin{itemize}
    \item Conhecer o público-alvo;
    \item Definir o propósito e os objetivos;
    \item Selecionar e preparar os dados relevantes;
    \item Focar na clareza e eficácia da comunicação;
\end{itemize}
\end{comment}

O processo de desenvolvimento de \textit{dashboards} é um processo iterativo, onde a constante comunicação entre os seus intervenientes é fundamental para garantir que o produto final satisfaça as necessidades e expectativas do público-alvo. Este diálogo é muitas vezes atrasado, e por vezes omitido, uma vez que existe uma falta de ferramentas que suportem esta fase inicial de design e prototipagem. Esta falta de comunicação pode resultar numa implementação falhada, onde os \textit{dashboards} não satisfazem as necessidades do seu público-alvo, o que resulta num desperdício de tempo e recursos.

Neste contexto, surge a necessidade de uma ferramenta que suporte de forma eficaz todo o processo de conceção e prototipagem de \textit{dashboards} interativos, uma vez que atualmente não existe nenhuma ferramenta de suporte que permita aos seus \textit{designers} demonstrar aos utilizadores, de forma visual e interativa, como será o produto final do \textit{dashboard}, nomeadamente, que visualizações irão ser usadas, qual a estrutura e organização optada, que interações vão estar disponíveis e quais os seus impactos. A linguagem \gls{IVML} foi criada com o principal intuito de preencher essa lacuna, oferecendo um modelo gráfico capaz de representar as diferentes componentes que integram um \textit{dashboard}. No entanto, como é referido na secção \ref{sec:cont_e_motiv}, a \gls{IVML} por si só não assegura a execução de um processo iterativo de prototipagem, uma vez que necessita de dispor uma ferramenta que suporte a sua implementação e utilização. A ferramenta desenvolvida deve ser de fácil utilização, nomeadamente na definição do seu \textit{layout}, na definição das visualizações mais apropriadas, e nas suas componente de navegação, interação e da recolha de \textit{feedback} visual destas. A ferramente deverá possibilitar a simulação do comportamento do \textit{dashboard}, o que torna a recolha de \textit{feedback} mais natural e intuitiva. Por fim, a ferramente deve permitir a exploração de alternativas, tanto a nível do seu \textit{layout} como das visualizações escolhidas, permitindo a inclusão de imagens de visualizações já implementadas com dados reais de forma a assimilar o comportamento do \textit{dashboard} final.

\section{Abordagem e contributos esperados}
\label{sec:contribuicoes}

% Basear me no documento de resumo que fiz sobre a reunião de dia 05/27/2025
% IVML-em detalhe, Prototipos-não tanto em detalhe, casos de uso
% No final relatar o porquê da minha abordagem ser importante na vida das dashboards
O estudo realizado na presente dissertação parte da identificação da necessidade de uma ferramenta que apoie de forma direta o processo de design e prototipagem de \textit{dashboards}. Ao contrário de propostas anteriores - mais focadas na implementação de ferramentas que representassem diretamente a linguagem \gls{IVML} -, a abordagem proposta propõe uma estratégia metodológica mais abrangente.

O ponto de partida começa pela definição e implementação de um meta-modelo capaz de representar os diferentes elementos que compõem um \textit{dashboard}, bem como todo o processo que o envolve, desde a conceção inicial onde são definidos os objetivos e âmbitos de análise, até à implementação final do \textit{dashboard}, onde são documentados todos os seus elementos e interações presentes. Este meta-modelo vai servir como suporte, não só para a solução desenvolvida, mas também para a contrução de uma metodologia que permita documentar todo o processo de prototipagem de \textit{dashboards}.

Para além do desenvolvimento de um meta-modelo, prevê-se a implementação de uma ferramenta \textit{web-based}, o \textit{Dashboard Designer}, que irá servir de suporte à linguagem \gls{IVML} e às necessidades do utilizador, permitindo simular e documentar todas as fases de criação de um \textit{dashboard}. Este processo irá ser inicializado através da criação de protótipos, de forma a simular a aparência e as funcionalidades finais pretendidas. Esta ferramenta irá ser fundamentada segundo príncipios de \gls{HCI}.

Por fim, irão ser implementados e testados casos de uso reais, através da realização de entrevistas e testes com diferentes \textit{designers}, com o principal intuito de validar a aplicabilidade e eficácia da solução desenvolvida. 

\section{Organização do documento}
\label{sec:organizacao}

Neste capítulo irá ser descrita a organização do documento de forma a facilitar a leitura e compreensão do mesmo. 

No capítulo atual é feita uma contextualização do tema da dissertação, dando ênfase à crescente necessidade de ferramentas para o design de \textit{dashboards} interativos (\ref{sec:cont_e_motiv}). É identificado e apresentado o problema atual no desenvolvimento de \textit{dashboards}, refletido pela falta de ferramentas de apoio à prototipagem e comunicação entre os seus utilizadores (\ref{sec:des_problema}). Por fim, é apresentada a abordagem ao problema identificado bem como os contributos esperados (\ref{sec:contribuicoes}).

No capítulo \ref{cha:conceitos_base} são abordados os conceitos teóricos que suportam esta dissertação. O capítulo é iniciado pela apresentação dos conceitos relacionados com a visualização de dados (\ref{sec:vis_dados}), abordando os seus princípios (\ref{sub:fundamentos}), o desenho de sistemas de visualização interativa de dados (\ref{sub:desenho_sistemas_vis}), o papel que a interação desepenha neste tipo de sistemas (\ref{sub:elem_int}), terminando pela apresentação das diferentes taxonomias de interação que suportam a visualização de dados (\ref{sub:taxonomias_interacao}). Na secção seguinte, é feita uma introdução ao design de \textit{dashboards} (\ref{sec:design_dashboard}), onde são abordados os diferentes tipos de \textit{dashboards} e as suas caracteristicas (\ref{sub:tipos_dashboards}), as recomendações e boas práticas a ter em consideração durante o seu desenvolvimento (\ref{sub:recomendacoes_design}), os desafios e limitações no design de \textit{dashboards} (\ref{sub:desafios_design}), terminando pela listagem e análise das diferentes ferramentas de desenho de \textit{dashboards} existentes (\ref{sub:ferramentas}). De forma semelhante, na secção seguinte é feita uma introdução ao design de interfaces de utilizador (\ref{sec:ui_design}), onde são introduzidos os principios de usabilidade e experiência do utilizador em interfaces interativas. Na secção seguinte é introduzido o conceito de avaliação da usabilidade de linguagens visuais (\ref{sec:avaliacao_usabilidade}), onde é abordado o conceito teórico \textit{Physics of Notation} e os princípios que suportam esta teoria descritiva. Por fim, é realiaza a descrição técnica da linguagem IVML (\ref{sec:esp_ivml}).

No capítulo \ref{cha:dashboard_designer} vai ser apresentado o Dashboard Designer, começando por explicar o seu processo de abordagem (\ref{sec:dashboard_abordagem}), passando à exibição de forma detalhada o meta-modelo desenvolvido de suporte à ferramenta que irá ser desenvolvida (\ref{sec:meta_modelo}), terminando com uma breve menção aos protótipos realizados numa fase inicial da dissertação (\ref{sec:prototipo_preliminar}).

No capítulo \ref{cha:abordagem_plano} é apresentada a proposta de abordagem para a segunda fase da dissertação, onde são apresentadas as tecnologias e processo de desenvolvimento que serão utilizados (\ref{sec:tecnologias_processo}), bem como o plano de trabalho que irá ser seguido ao longo da sua implementação (\ref{sec:plano_trabalho}). 

%Nocapítulo 3 é apresentado o IVML Designer. Começa-se por mostrar o meta-modelo desenvolvido para a linguagem IVML, especificando quais os elementos que irão fazer parte da ferramenta bem como todas as decisões tomadas durante o período anterior ao desenvolvimento.